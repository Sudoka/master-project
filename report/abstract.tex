\chapter*{Abstract}
\addcontentsline{toc}{chapter}{Abstract}

In this report, I present an integration of a road generation algorithm, described in \cite{roadgen}, with the GPU accelerated snow simulator developed at the HPC lab at the Department of Computer and Information Science at the Norwegian University of Science and Technology (NTNU). I also present a USGS DEM to RAW height map converter, for using real-world maps in the snow simulator. 

Given a height map of a terrain, my road generation application generates a road that minimizes a cost function that depends on the length, slope and curvature of the road. This is accomplished using an optimized A* search, generating a discrete shortest path represented by a series of nodes. 

The road trajectory is computed by using the nodes of the discrete shortest path as control points in a clothoid spline, which has a curvature that varies linearly with the length of the curve. The trajectory is used to generate a RoadXML file, which is a description of the road (trajectory, road profile, textures, etc.); this RoadXML file is then used in SCANeR Studio developed by Octal\cite{octalstudio} to generate a triangle mesh of the road. 

The mesh generated by SCANeR Studio is stored in the Wavefront OBJ format, and this file is read to load the mesh into the snow simulator for rendering. Materials defined in the OBJ file is also loaded. The terrain is adjusted to accommodate the road model as necessary, by constructing a distance map for each vertex in the road mesh, then adjusting the terrain elevation according to the closest vertices.

I present visual results showing the roads as they appear in the terrain. I also present performance results where different grid resolutions are tested; these results show that for many terrains, a lower resolution may not give a much higher cost of the trajectory, while giving much lower running time on searches. In addition to this, I present performance results on using Dijkstra's algorithm vs an A* search for generating the roads. The results show a clear difference, A* being up to 2.3 times faster, averaging at 1.7 times faster using the four test maps.

