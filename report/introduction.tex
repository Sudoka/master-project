\chapter{Introduction}
\label{chap:introduction}
In this report, I present my work of integrating procedurally generated roads with a snow simulator developed at the High Performance Computing lab at NTNU. I implement the road generation algorithm from \cite{roadgen}, output a RoadXML file\cite{roadxml}, use an external tool to generate a road mesh, and finally import this mesh into the snow simulator. 

The first part of this project, generating the road trajectory, is done using an A* graph search with points on a height map as the nodes, i.e. we perform a discrete shortest path search over the (potentially) continous domain that is the terrain. Then, in order to generate a RoadXML file, the actual trajectory is computed from the discrete shortest path by using the nodes along the path as control points for a clothoid spline. The clothoid spline is constructed using the method described in \ref{clothoid}.

A road mesh is generated using SCANeR Studio from Octal, by importing the RoadXML file. A triangle mesh is output in the Wavefront OBJ format, which is then parsed by the snow simulator. Finally, the terrain is adjusted so it will match the road model; this is neccesary due to the fact that the road trajectory elevation is sampled at discrete intervals when generating the RoadXML file, and that the road has a width which is not accounted for in the RoadXML.

\section{Project scope}
\label{sec:project-scope} 


\section{Report organization}
First, the report will go through some of the background material used in this project in chapter \ref{chap:background}; this includes solving shortest path problems, heuristic search, GPU programming and road generation. Then, the implementation of the road generator (which includes the A* search), integration with the snow simulator and the USGS DEM parser is presented in chapter \ref{chap:implementation}. Visual results as well as performance results are presented in chapter \ref{chap:results}. Finally, chapter \ref{chap:futurework} summarizes some of the possible improvements and future projects, and chapter \ref{chap:conclusion} concludes.
